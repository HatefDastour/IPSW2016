\documentclass[newPxFont,sthlmFooter]{beamer}

\include{Setup}
\include{Title}

\begin{document}
\maketitle

\begin{frame}{Motivation}
\begin{itemize}
\item There are two groups of Borrowers: Group 1 (Financial firms) and Group 2 (Oil and Gas); each group has a large number of firms (around 2000 and 9000). \justifying
\item Each borrower is ranked based on expert opinions. A rank is a number between 1 and 16 (default banks have rank 17). The bank has monthly data on ranks over a period of 10 years for each borrower. \justifying
\item If two borrowers belong to the same group, it is straightforward to compare the riskiness of these (by comparing the average scores for instance).
\end{itemize} 
\end{frame}

\begin{frame}
\begin{itemize}
\item Problem: If there are two borrowers coming from different groups, how can the bank know which borrower is riskier? \justifying
\item We then need to find a global ranking scale to compare the riskiness amongst borrowers from either groups.
\item Data: Group $A_i$ has $N_i$ firms, $i=1, 2.$ The ranks of borrower $a$ at time $t$ is $R^a_t, t=0,\dots T, R^a_t\in\{1,\dots 17\}.$
\end{itemize}
\end{frame}



\begin{frame}{First Approach}
\begin{itemize}
\item For borrower $a\in A_1,$ we define a local rank $\overline{R_t^a}$ by
$$\overline{R_t^a}=\sum_{t=1}^Tw_iR_t^a.$$
\item Supose that $(\overline{R_t^a})_{a\in A_1}$ are realizations of the same random variable $R^1$ whose distribution  function is $F_1.$
\item When $a$ varies in $A_1$, we can plot a histogram for $F_1.$
\end{itemize}
\end{frame}

\begin{frame}
\begin{itemize}
\item We then approximate the discrete empirical distribution $F_1$ by a smooth well-known distribution function $\hat{F_1}$ depending on the data.
\item We define the conditional probability of default 
 
$$PD^1(x):=PD(R^1=D|R^1=x)=\frac{1-\hat{F_1}(D)}{1-\hat{F_1}(x)}.$$
\item For two arbitrary borrowers $a, b$: we compare $PD^1(\overline{R_t^a})$ and $PD^2(\overline{R_t^b})$. 
\end{itemize}
\end{frame}

\begin{frame}{Second Approach}
\begin{itemize}
\item In each group, we suppose that borrowers are realizations of the same markov process ($R^1$ for group 1 and $R^2$ for group 2).
\item From the data, we get estimates for the  transition matrices $A_1$ and $A_2.$
\item Suppose that a borrower $a\in A_1$ has current rank $R^a_t=x,$ we can calculate the probability of default in a time horizon $\tau$ by
$$PD^1(x):=PD(R^1_s=D, s=t,\dots t+\tau|R_t=x).$$
\end{itemize}
\end{frame}
\begin{frame}
\begin{itemize}
\item Given two borrowers $a\in A_1, b\in A_2$ with current ranks $x, y$. We compare $PD^1(x)$ and $PD^2(y).$
\end{itemize}
\end{frame}
%\begin{frame}{Comparison}
%
%\end{frame}
%
%\begin{frame}{Conclusion}
%
%\end{frame}


\end{document} 