%%%%%%%%%%%%%%%%%%%%%%%%%%%%%%%%%%%%%%%%%%%%%%%%%%%%%%%%%%%%%%%%%%%
%
% Ce gabarit peu servir autant les philosophes que les scientifiques ; 
% et même d'autres genres, vous en faites ce que vous voulez.
% J'ai modifié et partagé ce gabarit afin d'épargner à d'autres 
% d'interminables heures à modifier des gabarits d'articles anglais. 
% 
% L'ajout d'une table des matières et une bibliographie a été ajoutée,
% rendant le gabarit plus ajusté aux besoins de plusieurs.
%
% Pour retrouvé le gabarit original, veuillez télécharger les
% documents suivants: llncs2e.zip (.cls et autres) et 
% typeinst.zip (.tex). Les documents ci-haut mentionnés ne sont pas 
% disponibles au même endroit, alors je vous invite à fouiller le web. 
%
% Pour l'instant (02-2016) ils sont disponibles tous deux ici :
%
% http://kawahara.ca/springer-lncs-latex-template/
%
% Netkompt
%
%%%%%%%%%%%%%%%%%%%%%%%%%%%%%%%%%%%%%%%%%%%%%%%%%%%%%%%%%%%%%%%%%%%


%%%%%%%%%%%%%%%%%%%%%%% file typeinst.tex %%%%%%%%%%%%%%%%%%%%%%%%%
%
% This is the LaTeX source for the instructions to authors using
% the LaTeX document class 'llncs.cls' for contributions to
% the Lecture Notes in Computer Sciences series.
% http://www.springer.com/lncs       Springer Heidelberg 2006/05/04
%
% It may be used as a template for your own input - copy it
% to a new file with a new name and use it as the basis
% for your article.
%
% NB: the document class 'llncs' has its own and detailed documentation, see
% ftp://ftp.springer.de/data/pubftp/pub/tex/latex/llncs/latex2e/llncsdoc.pdf
%
%%%%%%%%%%%%%%%%%%%%%%%%%%%%%%%%%%%%%%%%%%%%%%%%%%%%%%%%%%%%%%%%%%%

\documentclass[runningheads,a4paper]{llncs}

\usepackage[utf8]{inputenc}

\usepackage{natbib}
\bibliographystyle{apalike-fr}

\usepackage{amssymb}
\setcounter{tocdepth}{3}
\usepackage{graphicx}

%\usepackage[french]{babel} % Pour adopter les règles de typographie française
\usepackage[T1]{fontenc} % Pour que les lettres accentuées soient reconnues

\usepackage{url}
\urldef{\mailsa}\path|{alfred.hofmann, ursula.barth, ingrid.haas, frank.holzwarth,|
\urldef{\mailsb}\path|anna.kramer, leonie.kunz, christine.reiss, nicole.sator,|
\urldef{\mailsc}\path|erika.siebert-cole, peter.strasser, lncs}@springer.com|    
\newcommand{\keywords}[1]{\par\addvspace\baselineskip
\noindent\keywordname\enspace\ignorespaces#1}

\begin{document}

\mainmatter 

\title{How to compare two relative rankings?}

\titlerunning{}

\author{
Hatef Dastour (U. Calgary), Qiwei Feng (U. Manitoba), \\
Neal Madras (York U.), 
Weifei Ouyang (Shanghai Jiao Tong U.),
Negin Pasban Roozbahani (U. Manitoba), 
Alessandro Selvitella (McMaster U.)\\ 
Tuan Tran (McMaster U.)
}

\institute{}

\authorrunning{}

\toctitle{Abstract}
\tocauthor{{}}

\maketitle

\begin{abstract}
This work proposes two methods to compare the creditworthiness of two arbitrary borrowers coming from two different groups: Financial Firms and  Oil \& Gas Companies. The first method relies on a static measure of Distance to Default, and the second one is based on modelling the credit score as a Markovian process.
\end{abstract}

\medskip

\begingroup
\let\clearpage\relax
\tableofcontents
\addcontentsline{toc}{section}{Introduction}
\endgroup

\medskip
\medskip

\section*{Introduction}
Credit ranking plays an important role in banking industry. Banks rely on credit scores of their borrowers rated by rating agencies to decide whether or not they lend money to these  borrowers. When borrowers come from the same industry, it is straightforward to compare their creditworthiness based on their credit scores. However, if two borrowers come from two  different groups, it is no longer straightforward just to compare their scores because each group has its own industry-specific characteristics. The question is how can we come up with a global measure in order to normalize the different relative rankings.

In this project, we work on a dataset with two groups of firms: the first one consists of firms coming from financial sector (non banks), and the second one consists of firms  coming form Oil\& Gas industry, each group has a large number of firms (around 2000 and 9000 respectively). Rank of firms is any integer number between 1 to 17, where 1 corresponds to the best credit quality and 17 corresponds to default firms. Firms are rated based on experts' opinion. We suppose that ranks are relatively accurate and try to convert them into some sort of global measure. 

We propose in this work two different measures for creditworthiness: the first one relies on a static measure of Distance to Default, and the second one is based on modelling the credit igration as a markovian process. 


\section{First Approach}

We need some notations: Suppose that there are two groups of firms: Group $A_i$ has $N_i$ firms, $i=1, 2.$ The ranks of borrower $a$ at time $t$ is $R^a_t, t=0,\dots T, R^a_t\in\{1,\dots 17\}.$

The idea is the following:
\begin{itemize}
    \item Fix a borrower $a$ in one group, say $a\in A_1.$ We come up with an static  score for $a.$ The easiest way to do this is to take arithmetic mean or weighted average with power weights.
    $$\overline{R^a}=\sum_{t=1}^Tw_tR_t^a.$$
    Here,  we can chose $w_t=\frac{1}{T}$ or $w_t=\frac{\rho^{t-1}}{1-\rho^T}$ for some $\rho \in (0, 1).$ This measure is still a local one as it does not take into account any industry-related information.
    \item Now, in each group, we suppose that firms bear common information about the group. By this, it is reasonable to suppose that $(\overline{R^a})_{a\in A_1}$ are drawn from the same distribution function $F_1$ of some random variable $M_1.$ We can build an empirical distribution for $F_1$ based on $(\overline{R^a})_{a\in A_1}$.
    \item Now, if a given borrower comes from group $A_1$ with local score $\overline{R^a}=x,$ then we define the Relative Distance to Default  as
    $$RDD^1(x)=P(R^1>16|R_1\ge x)=\frac{P(R^1>16)}{P(R^1\ge x)}.$$
    \item We repeat the previous procedure for borrowers from group 2. Now if a borrower from Group 1 with local score $x$ and another borrower from Group 2 with local score $y$, the bank has to compare $RDD^1(x)$ and $RDD^2(y)$ to decide which borrower is riskier to lend money.
    
    \item We can go a step further by converting credit scores from Group 1 into those in Group 2. More precisely, we find an creasing  mapping $\phi: \{1,\dots,17\}\to [1, 17]$ that matches the conditional probability of default, i.e. we find $\phi$ in such a way that
    $$RDD^1(x)=RDD^2(\phi(x)).$$
    
\end{itemize}


\section{Second Approach}
In this approach, we suppose that the credit profiles of all borrowers in the same group (say Group 1) are realizations of the same homogeneous Markov process $M_1$ of constant transition matrix $Q$ and constant generator matrix $\Lambda.$ By definition $Q_{ij}$ denotes the probability that $M_1$ is migrated from  rank $i$ to rank $j$ after one year. It is a well-known fact that $Q=e^{\Lambda}$. 

The idea of this method is the following
\begin{itemize}
    \item First, we estimate the matrices $Q$ (or equivalently $\Lambda$) form the data. There are two ways of doing this. The first one is to estimate the one-month step  transition matrix  directly from the data. The formula we use is the following
    $$Q^{1/12}_{ij}=\frac{N_{ij}[0, T]}{\sum_1^TN_i[t]},$$
    where $N_{ij}[0, T]$ denotes the number of migration from state $i$ to state $j$ during the period $[0, T]$ (here $T=127$ months), and $N_i[t]$ denotes how many firms whose rank is $i$ at time $t.$ Finally, the one-year transition matrix is given by $Q=[Q^{1/12}]^{12}.$
    
    The second way to estimate $Q$ is to estimate the generator matrix $\Lambda$ first and then $Q$ is given by $e^{\Lambda}.$ To estimate $\Lambda,$ we follow Lando and Skodeberg (2012). We have $\Lambda_{ii}=-\sum_{j\not=i}\Lambda_{ij}.$ and
    $$\Lambda_{ij}:=12\frac{N_{ij}[0, T]}{\sum_1^TN_i[t]}.$$
    The only advantage of the second estimation is that it can help deal with  the case when there are many zeros in the estimated matrix $Q.$
    
    \item Once we have estimate of $\Lambda$, The $\tau-$ year transition matrix is given by $Q^{tau}=e^{\Lambda \tau}.$ $\tau$ is determined by the bank, typically one year.
    \item Now, if a borrower from Group 1 has current rank of $x$, its probability of default in $\tau-$year will be $Q^{tau}_{x,17}.$
    \item Finally, in order to compare the creditworthiness of two borrowers, one from Group 1 with current score x and one  from Group 2 with score y, the bank only needs to compare $Q^{\tau}_{x,17}$ and $P^{\tau}_{y,17},$ where $P$ denotes the transition matrix of Group 2 estimated in the same way for $Q.$
\end{itemize}

Another possible way to compare the creditworthiness of two borrowers with current ranks $x, y$ (comming from Group 1 and Group 2, respectively) is to compare the two first passage times when the transition probabilities from these states to default excess certain threshold $p_0$ (for instance $p_0=0.5$). This ``half-life''
method can be considered as a dual one to the previous method, it gives more information about necessary time  for the default event to  be likely to happen. More precisely, we define
$$N_x=\inf\{n: Q^{tau}_{x,17}\ge p_0\}$$
and
$$N_y=\inf\{n: P^{tau}_{y,17}\ge p_0\}$$
If $N_x>N_y$ then we can conclude that rank $x$ in Group 1 is less risky than rank $y$ in Group 2 with default level $p_0.$

%\medskip
Results of the Markov chain analysis were somewhat mixed.  The one-year default probabilities
were comparable between the two groups for ranks 8 through 15;  the probabilities for
rank $k$ in Oil and Gas roughly corresponded to rank $k+1$ in Financial Services.
The better ranks had very low default probabilities, which were harder to compare.  
The ``half-life''  method gave a different picture, suggesting that all but the weakest ranks 
of Financial companies took significantly longer to default than most of the Oil and Gas 
companies.


\bibliography{}
\nocite{*} 

\end{document}
