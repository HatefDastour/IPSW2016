\documentclass{beamer}


\begin{document}

\frame{
\underline{Markov Chain approach}

For each company $a$, we model the random process $R^a_t$ ($t=1,\ldots,T$) as a 
Markov chain on the state space $\{1,2,\ldots,16,D\}$.  Assume that there is one transition 
probability matrix for the Oil \& Gas companies, and another for the Financial Services companies.

\medskip

For states $i$ and $j$, we estimate the transition probability $P(i,j)$ by 
\[    \frac{\hbox{number of observed transitions from $i$ to $j$}}{\hbox{number of 
       observations in state $i$}}
\]
(within each sector).

\medskip

\textbf{Remarks}:
\\  (1)  We see some transitions $D\rightarrow i$ (e.g.\ companies emerging from
bankruptcy protection), but we ignore these and treat $D$ as an absorbing state (i.e., $P(D,D)=1$).
\\
(2)  There are no 16's in the Financial sector.  There are a few 16's in Oil \& Gas, but we 
never see them change.  So we ignore state 16.
}

\frame{
Given a transition probability matrix $P$, there are different ways to assess default probabilities of 
states for comparison purposes.

\medskip
$\bullet$  \underline{Single month}:  Look at $P(i,D)$ for each $i$.  These numbers are small 
(sometimes 0), with substantial statistical uncertainty (via the Clopper-Pearson confidence intervals
for binomial distributions).

\smallskip
$\bullet$ \underline{One year}:  Compute $P^{12}$, the 12-month transition matrix, and look at 
$P^{(12)}(i,D)$ for each $i$.

\smallskip
$\bullet$ \underline{``Half-life''}:  Starting in state $i$, let $H(i)$ be the number of periods needed for 
the probability of default to exceed 0.5:
\[    H(i) \,:=\, \min\{t:  P^{(t)}(i,D) \,\geq \,0.5\}.   \]
 
}

\end{document}


\frame
{
  \frametitle{Features of the Beamer Class}

  \begin{itemize}
  \item<1-> Normal LaTeX class.
  \item<2-> Easy overlays.
  \item<3-> No external programs needed.      
  \end{itemize}
}

